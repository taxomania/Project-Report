\chapter{Evaluation}
\label{cha:eval}
With implementation and analysis complete, one can now identify and evaluate the key successes and shortcomings of this project. These evaluations have been performed on the basis of the following questions and has been carried out by means of independent user evaluations of the software.

\begin{itemize}
\item Does the system work?
\item Is the information found novel and interesting?
\item Is the system easy to use?
\end{itemize}

The system was shown to work to an accuracy of 73\% when measuring against software names. This is a relatively low performance rating as over larger sets of data the number of inaccurate results will be much greater. This is mainly down to the recall of the system. There are many situations where references to software have been missed in tweets. There is also a large set of tweets where they are in fact not related to software or explicity mention their names. However this may be caused by the ambiguous nature of the terms in the dictionary and keyword database. Improving these results may require more linguistic rules to better analyse text.

The information found by the system is mainly sentiment data. While this may be interesting to certain parties, user opinion appears somewhat indifferent. Sentiment data would be interesting to the developers, and users looking for new software. However, software is not explicitly flagged as new when it first appears in tweets. Also, unless a user specifically searches for a new piece of software, it is unlikely to appear in many tweets present in the database. On the other hand, with more popular software, the sentiment information may not be particularly interesting to most users because the general perception towards it is likely to be common knowledge.

Finally, the system's usability is one of the key issues in this project. The web application has a very simple, straightforward interface. Some users found this to be more appealing as looks tidy, however, others felt it was minimalistic, though easy to use. Users also suggested the analysis page required more charts and extra information, such as the tweets themselves.


