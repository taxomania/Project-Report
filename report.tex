\documentclass[12pt]{third-rep}

%% Any characters from a % to the end of line are comments.

%% The third-rep class and this starter kit were written by 
%% Graham Gough <graham@cs.man.ac.uk>
%% If you have any comments or questions regarding this document,
%% please post them to the local newsgroup man.cs.tex.

%% This skeleton report is organised as a master file called
%% report.tex which then includes files for individual parts including
%% abstract.tex, chapter1.tex, chapter2.tex, chapter3.tex and
%% appendix1.tex.  

%% The third-rep style is a locally created style based on the
%% standard LaTeX report style. If you really want to have a look at
%% it, its source can be found in
%% /usr/local/share/texmf/tex/latex/mancs/third-rep.cls
%%
%% More information about LaTeX in general and the local setup in
%% particular can be found on the web at 
%% http://www.cs.manchester.ac.uk/software/tex/
%%
%%%%%%%%%%%%%%%%%%%%%%%%%%%%%%%%%%%%%%%%%%%%%%%%%%%%%%%%%%%%%%%%%%%%%%%%
%%
%% This is an example of how you load extra packages.
%% Some packages are already loaded in the third-rep class

\usepackage{url} % typeset URL's sensibly
\usepackage{amsmath}
\usepackage{pslatex} % Use Postscript fonts
\usepackage[utf8]{inputenc}
\usepackage[T1]{fontenc}
\usepackage{listings}
\usepackage{color}
\usepackage{algpseudocode}

% Universal listings settings 
\lstset{ %
  basicstyle=\ttfamily\footnotesize,       % the size of the fonts that are used for the code
%  numbers=left,                  % where to put the line-numbers
%  numberstyle=\tiny\color{black},% the style that is used for the line-numbers
%  stepnumber=2,                  % the step between two line-numbers. If it's 1, each line 
                                  % will be numbered
%  numbersep=5pt,                 % how far the line-numbers are from the code
  backgroundcolor=\color{white},  % choose the background color. You must add \usepackage{color}
  showspaces=false,               % show spaces adding particular underscores
  showstringspaces=false,         % underline spaces within strings
  showtabs=false,                 % show tabs within strings adding particular underscores
  frame=single,                   % adds a frame around the code
  rulecolor=\color{black},        % if not set, the frame-color may be changed on line-breaks within not-black text (e.g. commens (green here))
  captionpos=b,                   % sets the caption-position to bottom
  breaklines=true,                % sets automatic line breaking
  breakatwhitespace=false,        % sets if automatic breaks should only happen at whitespace
  tabsize=4,
  title=\lstname                  % show the filename of files included with \lstinputlisting;
}

\definecolor{javared}{rgb}{0.6,0,0} % for strings
\definecolor{javagreen}{rgb}{0.25,0.5,0.35} % comments
\definecolor{javapurple}{rgb}{0.5,0,0.35} % keywords
\definecolor{javadocblue}{rgb}{0.25,0.35,0.75} % javadoc


\newcommand{\java}{
% Java specifics
\lstset{
  language=Java,
  keywordstyle=\color{javapurple}\bfseries,
  stringstyle=\color{javared},
  commentstyle=\color{javagreen},
  morecomment=[s][\color{javadocblue}]{/**}{*/},
}}

\newcommand{\python}{
\lstset{language=Python,
  keywordstyle=\color{black}\bfseries,
  stringstyle=\color{black},
  commentstyle=\color{black}}}
%%  The best way to latex just one chapter is to uncomment lines such as
%% the next:
%\includeonly{chapter1}

%% This defines the title (the \\ forces a line break)
\title{Text Mining Twitter for Software\\
  and User Perception}
%% and author
\author{Tariq Patel}
%% and supervisor
\supervisor{Dr. Goran Nenadic}
%% and the year of the report
\reportyear{2012}

%% this defines the file that contains the text of the abstract, there
%% must be one of these by the time you submit your report.
\abstractfile{abstract.tex}

%% this defines the file that contains the acknowledgements (it can be
%% omitted if you don't feel like thanking anyone
\thanksfile{merci.tex}

%% End of preamble, the actual document starts here
\begin{document}

%% This actually creates the title and abstract pages
\dotitleandabstract

%% Generate contents etc
\tableofcontents
\listoffigures
\listoftables
\lstlistoflistings

%% These include the actual text
\include{chapter*}


\nocite{*}
\bibliography{refs}             % this causes the references to be
                                % listed

\bibliographystyle{alpha}       % this determines the style in which
                                % the references are printed, other
                                % possible values are plain and abbrv
%% Appendices start here
\appendix
\chapter{Dictionary of Software and Keywords}
\label{app:dictionary}

\begin{table}[h]
\begin{center}
\begin{tabular}{|l|l|l|}\hline
software&app&mac\\\hline
pc&alpha&beta\\\hline
version&source code&sdk\\\hline
release&api&version\\\hline
game&&\\\hline
\end{tabular}
\end{center}
\caption{List of keywords}
\end{table}

\begin{table}[h]
\begin{center}
\begin{tabular}{|l|l|l|}\hline
Apache&Mozilla&Activision\\\hline
Infinity Ward&Blizzard&EA\\\hline
Rockstar Games&HP&Codemasters\\\hline
Valve&Adobe&Blackberry\\\hline
Cisco&SlingMedia&Maxis\\\hline
Rarlabs&VideoLan&ATI\\\hline
Atari&Symantec&Nvidia\\\hline
Oracle&Apple&Microsoft\\\hline
Google&&\\\hline
\end{tabular}
\end{center}
\caption{Dictionary of companies}
\end{table}

\begin{table}[h]
\begin{center}
\begin{tabular}{|l|l|l|}\hline
WebOS&Android&Bada\\\hline
Symbian&Linux&Fedora\\\hline
Ubuntu&Windows&iOS\\\hline
OS X&MacOS&FreeBSD\\\hline
\end{tabular}
\end{center}
\caption{Dictionary of operating systems}
\end{table}


\begin{table}
\begin{center}
\begin{tabular}{|l|l|l|}\hline
ZeuAPP&Bitcoin&vtiger CRM\\\hline
ReOS&SugarCRM&OrangeHRM\\\hline
Ebase&Dolibarr ERP/CRM&Bonita Open Solution\\\hline
Adempiere&bookyt&Compiere\\\hline
FrontAccounting&GnuCash&Grisbi\\\hline
HomeBank&jFin&JFire\\\hline
JGnash&JQuantLib&KMyMoney\\\hline
LedgerSMB&MibianLib&Mifos\\\hline
Octopus Micro Finance Suite&Openbravo&OpenERP\\\hline
Postbooks&QuickFIX&QuickFIX/J\\\hline
SQL Ledger&Tryton&TurboCASH\\\hline
WebERP&refbase&Koha\\\hline
NewGenLib&OpenBiblio&PMB\\\hline
SimPy&CellProfiler&ImageJ\\\hline
Endrov&Jmol&Molekel\\\hline
MeshLab&PyMOL&QuteMol\\\hline
RasMol&Avogadro&Ascalaph Designer\\\hline
GROMACS&LAMMPS&MDynaMix\\\hline
NAMD&Chemistry Development Kit&JOELib\\\hline
OpenBabel&P-GRADE Portal&OpenCog\\\hline
OpenCV&AForge.NET&ROS\\\hline
TREX&CMU Sphinx&Emacspeak\\\hline
Festival Speech Synthesis System&NVDA&Text2Speech\\\hline
NonVisual Desktop Access&ESpeak&Dasher\\\hline
Gnopernicus&Virtual Magnifying Glass&OpenAFS\\\hline
Eucalyptus&AppScale&FusionCharts\\\hline
ParaView&Orange&RapidMiner\\\hline
Scriptella ETL&Weka&jHepWork\\\hline
Konstanz Information Miner&KNIME&ELKI\\\hline
Lucene&Solr&Xapian\\\hline
Nutch&CloverETL&Talend\\\hline
Pentaho&SpagoBI&Limesurvey\\\hline
OpenX&RSS Bandit&RSSOwl\\\hline
Akregator&Liferea&Ekiga\\\hline
\end{tabular}
\end{center}
\caption{Dictionary of software}
\end{table}
\begin{table}
\begin{center}
\begin{tabular}{|l|l|l|}\hline
FreePBX&FreeSWITCH&Jitsi\\\hline
QuteCom&sipX&Slrn\\\hline
FreeNX&OpenVPN&rdesktop\\\hline
VNC&RealVNC&TightVNC\\\hline
UltraVNC&xrdp&HTTrack\\\hline
Wget&Apache Cocoon&AWStats\\\hline
BookmarkSync&CougarXML&curl-loader\\\hline
HTTP File Server&Distributed ICDL Crawler&Crawley Framework\\\hline
lighttpd&nginx&NetKernel\\\hline
Piwik&Qcodo&Web-Developer Server Suite\\\hline
XAMPP&Zope&Oxwall\\\hline
Liferay&Sun Java System Portal Server&uPortal\\\hline
Apache Axis2&Apache Geronimo&GlassFish\\\hline
CORBA&JacORB&Jakarta Tomcat\\\hline
JBoss&ObjectWeb JOnAS&OpenSplice DDS\\\hline
SmartVariables&OpenLDAP&JXplorer\\\hline
openVXI&YaCy&Gnaural\\\hline
DoceboLMS&eFront&GCompris\\\hline
IUP&Moodle&Omeka\\\hline
Sakai&Chamilo&openSIS\\\hline
ATutor&ILIAS&Kiten\\\hline
KVerbos&KTouch&KGeography\\\hline
KEduca&openlp.org&BibleDesktop\\\hline
BibleTime&Xiphos&SWORD Project\\\hline
SwordBible&Go Bible&jSword\\\hline
MacSword&Marcion&Eye of GNOME\\\hline
F-spot&Gqview&Gthumb\\\hline
imgSeek&Kphotoalbum&Dream DRM Receiver\\\hline
KToon&Synfig&K-3D\\\hline
OpenFX&Seamless3d&Pencil Animation\\\hline
SWFTools&Avidemux&AviSynth\\\hline
Blender&Cinelerra&CineFX\\\hline
Jahshaka&DScaler&DVD Flick\\\hline
\end{tabular}
\end{center}
\caption{Dictionary of software}
\end{table}
\begin{table}
\begin{center}
\begin{tabular}{|l|l|l|}\hline
DVDx&Kaltura&Kino\\\hline
Kdenlive&OpenShot Video Editor&PiTiVi\\\hline
VirtualDub&VirtualDubMod&Celtx\\\hline
KeePass&Password Safe&TeamLab\\\hline
Project.net&KAddressBook&Kontact\\\hline
KOrganizer&Novell Evolution&OpenSync\\\hline
Rachota Timetracker&Bugzilla&Mindquarry\\\hline
Redmine&Trac&Open Scene Graph\\\hline
OpenSCDP&CodeSynthesis XSD&CodeSynthesis XSD/e\\\hline
xmlbeansxx&Flex lexical analyser&Kodos\\\hline
phpCodeGenie&\^txt2regex\$&SableCC\\\hline
Autoconf&Automake&Xnee\\\hline
Memtest86&JSystem&GNU Debugger\\\hline
ClamAV&ClamWin&Gateway Anti-Virus\\\hline
AVG antivirus&Winpooch&MyDLP\\\hline
GnuPG&KGPG&Seahorse\\\hline
GnuTLS&OpenSSL&CrossCrypt\\\hline
FreeOTFE and FreeOTFE Explorer&Iptables&Coyote Linux\\\hline
Firestarter&IPFilter&ipfw\\\hline
IPCop&IPFire&M0n0wall\\\hline
PeerGuardian&pfSense&SmoothWall\\\hline
Shorewall&Untangle&Vyatta\\\hline
Zentyal&Lsh&OpenSSH\\\hline
PuTTY&Cyberduck&Agorum\\\hline
Anti-Spam SMTP Proxy&Balsa&Bogofilter\\\hline
Citadel/UX&Claws Mail&Dada Mail\\\hline
DSPAM&Enigmail&Exim\\\hline
Fdm&Fetchmail&FreePOPs\\\hline
FUDforum&Getmail&GNUMail\\\hline
Gnus&Gnuzilla&GPGMail\\\hline
GroupServer&Hypermail&I-sense\\\hline
IlohaMail&Internet Messaging Program&Libremail\\\hline
GNU Mailman&MailScanner&Mailx\\\hline
\end{tabular}
\end{center}
\caption{Dictionary of software}
\end{table}
\begin{table}
\begin{center}
\begin{tabular}{|l|l|l|}\hline
Majordomo&MH Message Handling System&MIMEDefang\\\hline
Movemail&Mozilla Mail&Mozilla Thunderbird\\\hline
Mutt&NeoMail&Open-Xchange\\\hline
POPFile&Qpopper&Roundcube\\\hline
Scalix&SimpleMail&Smail\\\hline
Smartlist&SpamAssassin&Spicebird\\\hline
SquirrelMail&Sylpheed&Sympa\\\hline
Uebimiau&Universal village collaboration suite&UW IMAP\\\hline
Vpopmail&Xuheki&Yet Another Mailer\\\hline
YPOPs!&Zarafa&Zimbra\\\hline
Adium&AMSN&Ayttm\\\hline
BitlBee&Bombus&Bombusmod\\\hline
Centericq&Climm&Coccinella\\\hline
Emesene&Fama IM&Gabber\\\hline
Gajim&Instantbird&JabberMixClient\\\hline
Jimm&JWChat&Kadu\\\hline
Kopete&Meetro&Miranda IM\\\hline
Monal&Naim&Neustradamus\\\hline
Pandion&Pidgin&Proteus\\\hline
Retroshare&SFLphone&Tapioca\\\hline
TerraIM&Tkabber&Tmsnc\\\hline
TorChat&Zephyr&uTorrent\\\hline
Skype&Sopcast&Spotify\\\hline
Google Chrome&Steam&TextMate\\\hline
Text Editor&TextEdit&nedit\\\hline
gedit&BitLord&bittorrent\\\hline
iTunes&AVG&MS Office\\\hline
f.lux&WinRar&VLC\\\hline
Radeon&Daemon tools&iTunes Match\\\hline
emacs&Firefox&Dreamweaver\\\hline
\end{tabular}
\end{center}
\caption{Dictionary of software}
\end{table}



\chapter{Bing API Integration}
\label{app:bing}
\lstinputlisting{bing.py}



\end{document}
