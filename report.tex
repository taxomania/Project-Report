\documentclass[12pt]{third-rep}

%% Any characters from a % to the end of line are comments.

%% The third-rep class and this starter kit were written by 
%% Graham Gough <graham@cs.man.ac.uk>
%% If you have any comments or questions regarding this document,
%% please post them to the local newsgroup man.cs.tex.

%% This skeleton report is organised as a master file called
%% report.tex which then includes files for individual parts including
%% abstract.tex, chapter1.tex, chapter2.tex, chapter3.tex and
%% appendix1.tex.  

%% The third-rep style is a locally created style based on the
%% standard LaTeX report style. If you really want to have a look at
%% it, its source can be found in
%% /usr/local/share/texmf/tex/latex/mancs/third-rep.cls
%%
%% More information about LaTeX in general and the local setup in
%% particular can be found on the web at 
%% http://www.cs.manchester.ac.uk/software/tex/
%%
%%%%%%%%%%%%%%%%%%%%%%%%%%%%%%%%%%%%%%%%%%%%%%%%%%%%%%%%%%%%%%%%%%%%%%%%
%%
%% This is an example of how you load extra packages.
%% Some packages are already loaded in the third-rep class

\usepackage{url} % typeset URL's sensibly
\usepackage{amsmath}
\usepackage{pslatex} % Use Postscript fonts
\usepackage[utf8]{inputenc}
\usepackage[T1]{fontenc}
\usepackage{listings}
\usepackage{color}
\usepackage{algpseudocode}

% Universal listings settings 
\lstset{ %
  basicstyle=\ttfamily\footnotesize,       % the size of the fonts that are used for the code
%  numbers=left,                  % where to put the line-numbers
%  numberstyle=\tiny\color{black},% the style that is used for the line-numbers
%  stepnumber=2,                  % the step between two line-numbers. If it's 1, each line 
                                  % will be numbered
%  numbersep=5pt,                 % how far the line-numbers are from the code
  backgroundcolor=\color{white},  % choose the background color. You must add \usepackage{color}
  showspaces=false,               % show spaces adding particular underscores
  showstringspaces=false,         % underline spaces within strings
  showtabs=false,                 % show tabs within strings adding particular underscores
  frame=single,                   % adds a frame around the code
  rulecolor=\color{black},        % if not set, the frame-color may be changed on line-breaks within not-black text (e.g. commens (green here))
  captionpos=b,                   % sets the caption-position to bottom
  breaklines=true,                % sets automatic line breaking
  breakatwhitespace=false,        % sets if automatic breaks should only happen at whitespace
  tabsize=4,
  title=\lstname                  % show the filename of files included with \lstinputlisting;
}

\definecolor{javared}{rgb}{0.6,0,0} % for strings
\definecolor{javagreen}{rgb}{0.25,0.5,0.35} % comments
\definecolor{javapurple}{rgb}{0.5,0,0.35} % keywords
\definecolor{javadocblue}{rgb}{0.25,0.35,0.75} % javadoc


\newcommand{\java}{
% Java specifics
\lstset{
  language=Java,
  keywordstyle=\color{javapurple}\bfseries,
  stringstyle=\color{javared},
  commentstyle=\color{javagreen},
  morecomment=[s][\color{javadocblue}]{/**}{*/},
}}

\newcommand{\python}{
\lstset{language=Python,
  keywordstyle=\color{black}\bfseries,
  stringstyle=\color{black},
  commentstyle=\color{black},
  morekeywords={as}}}
%%  The best way to latex just one chapter is to uncomment lines such as
%% the next:
%\includeonly{chapter1}

%% This defines the title (the \\ forces a line break)
\title{Text Mining Twitter for Software\\
  and User Perception}
%% and author
\author{Tariq Patel}
%% and supervisor
\supervisor{Dr. Goran Nenadic}
%% and the year of the report
\reportyear{2012}

%% this defines the file that contains the text of the abstract, there
%% must be one of these by the time you submit your report.
\abstractfile{abstract.tex}

%% this defines the file that contains the acknowledgements (it can be
%% omitted if you don't feel like thanking anyone
\thanksfile{merci.tex}

%% End of preamble, the actual document starts here
\begin{document}

%% This actually creates the title and abstract pages
\dotitleandabstract

%% Generate contents etc
\tableofcontents
\listoffigures
\listoftables
\lstlistoflistings

%% These include the actual text
\chapter{Introduction}
\label{cha:intro}
The success of a piece of software is based largely upon user opinion. Gathering such information is conventionally done through means of surveying groups of users. However, in the days of social media, people generally express their opinions on popular social networks or microblogging sites such as Facebook and Twitter. This means it is now much easier for companies to receive feedback on products they have developed by monitoring these networks.

\section{Aim and Motivation}
\label{sec:aim}
Twitter has been at the core of many data mining projects in recent years and this is due to the sheer amount of data produced on a daily basis. Twitter users now post in excess of 340 million tweets every day\cite{twitterblog} and as such Twitter provides a massive corpus\footnote{A collection of documents} for opinion mining and sentiment analysis.

By text mining Twitter posts for software, users are able to discover new tools or programs they have not come across before, as well as see reviews by other users.

Thus, the aim of this project is to develop a system that text mines Twitter posts to find software or software development tools that have been mentioned by its users and to discover the general sentiment of users towards these softwares.

\section{Challenges}
There are many challenges facing Natural Language Processing(NLP)-oriented projects. Table~\ref{challenges} shows the key issues to be faced in achieving the main aim.

With the millions of tweets posted every day on Twitter, one can safely presume that many of these will have no relevance to software or any of the other desired information. As such it will be vital to ensure only the most relevant tweets are extracted from Twitter for analysis so as not to waste resources.

Another issue is the world-wide nature of the Internet and microblogging networks like Twitter. This means that several tweets will not be in English and for this reason it would be more difficult to extract features from these tweets. To counter this, it will be necessary to filter tweets not only based on key words but also on their language.

A major issue in NLP research is that of text message shorthand. In a formal document this problem becomes somewhat irrelevant due to proper usage of Standard English. However, when working with the Twitter platform, the service's 140 character limitation on tweets means users are generally more likely to abbreviate their text and this allows for a lot of ambiguity in the context of each word, and variability in how users may say the same thing.

% EXAMPLES GO HERE

\begin{table}
\begin{center}
\begin{tabular}{|r|l|}\hline\hline
&Challenges\\\hline
1&Finding relevant tweets\\
2&Non-English tweets\\
3&Text message shorthand\\\hline\hline
\end{tabular}
\end{center}
\caption{Challenges to be faced in this project}\label{challenges}
\end{table}

\section{Objectives}
In order to successfully complete a project of this magnitude, the task at hand must be split into smaller steps.  These objectives are shown with their complexities and priorities in Table~\ref{objectives}.

\begin{table}
\begin{center}
\begin{tabular}{|r|c|c|c|}\hline\hline
&Task&Complexity&Priority\\\hline
1&Collect and filter tweets by keyword&Simple&High\\
2&Feature extraction&Complex&High\\
3&Analyse tweet sentiment&Intermediate&Medium\\
4&Structure and integrate data&Intermediate&High\\
5&Visualise data through GUI&Intermediate&Low\\
6&Evaluate the system&Complex&High\\\hline\hline
\end{tabular}
\end{center}
\caption{Complexity and priority of project objectives}\label{objectives}
\end{table}

\subsection{Collect and Filter Tweets by Keyword}
Collecting tweets is a core task in this project as all the work will be based on tweets stored in a database. Filtering through these is a relatively simple task in that it can be done using Twitter's APIs, but there are some complexities in ensuring they are all relevant.

The main idea at this stage is to collect tweets based on a set of key words and software names, programming languages, or company names stored in a dictionary, in order to retrieve relevant, software-related tweets.

\subsection{Feature Extraction}
Feature extracting is the core functionality set out to be achieved in this project. Using rule-based text mining techniques, the aim of this task will be to retrieve up to eight features from every tweet, which are shown in Table~\ref{features}.

\begin{table}
\begin{center}
\begin{tabular}{|r|l|}\hline\hline
&Feature\\\hline
1&Software name\\
2&Software version\\
3&Company or developer\\
4&Programming language\\
5&Operating system\\
6&Price\\
7&Relevant URLs\\
8&Tweet sentiment\\\hline\hline
\end{tabular}
\end{center}
\caption{Features to be extracted from tweets}\label{features}
\end{table}

These features have been selected in order to find useful information from tweets to be displayed to users. The \textbf{software name} is of course vital, in that this discovery is the main purpose of the project. The \textbf{version} of this software is important because major changes may have been made over the course of a few releases and so it is necessary to note which release people are referring to. The \textbf{company name} is not a major feature, however it may be interesting to know who developed a certain piece of software. It may also be used in a different scenario where a user of this system wishes to find public sentiment towards a company as opposed to some specific software. The \textbf{programming language} feature ideally signifies the language or languages in which the found software was developed in. However, as with the company field, this may be used to find sentiment towards specific programming languages or practices. The \textbf{operating system} field works in a similar fashion, in that its expected use is to find the operating systems upon which the found software runs, but it can also be used to find the sentiment towards a specific operating system. \textbf{Price} and \textbf{URL} extraction are geared towards retrieving information about the product for the user. The \textbf{tweet sentiment} aims to find the general sentiment towards a piece of software, and will be used in the aggregation process in the final stages when trying to establish public perception of the software.

\subsection{Analyse Tweet Sentiment}
Sentiment analysis is another of the more important tasks in this project. This is where tweets are analysed for subjectivity, i.e. whether the tweet is positive, negative or neutral, and this will be used to show the general user perception of each piece of software.

\subsection{Structure and Integrate Data}
The data needs to be structured and aggregated to be able to provide any meaningful output for the user. Without this step, the system is producing no useful information.

\subsection{Visualise Data Through GUI}
Visualising the data is a fairly low priority task in that the system first needs to gather the information. This project centres more around the core back-end development than user experience and as such only a simple user interface is needed in its initial stages.

\subsection{Evaluate the System}
The final evaluation of the produced system will be key in determining the success of this project. The system will be evaluated on the basis of the accuracy of retrieved results, the relevance and novelty of information, and general usability.

\section{Report Structure}
This report documents the implementation of a text mining system that is set out to achieve the previously stated goals. The remainder of this report has been split into 6 chapters.
Chapter~\ref{cha:background} details the general background of this project and previous work in the area.
Chapter~\ref{cha:design} goes into the design of the software implementation including use case analysis, the architecture of the system and the software engineering methodologies used.
Chapter~\ref{cha:impl} describes the process of implementing each stage of the project and goes into details of how specific aspects such as the Twitter API integration and feature extraction work.
Chapter~\ref{cha:results} illustrates the results and final outcomes of the project with any meaningful information gained.
Chapter~\ref{cha:eval} provides the general evaluation of the finished project, also outlining the successes and failures of the task at hand.
Finally, Chapter~\ref{cha:conclusion} concludes the author's conclusions of the project, with suggestions for further work.

\chapter{Background}
\label{cha:background}
This chapter provides an overview of the text mining field along with previous work in the area and all necessary background information required to understand the major tasks involved in this project.

\section{Text Mining}
\label{sec:textmining}
The information available in the world is growing exponentially, and the majority of this information(widely estimated at roughly 80\%) is unstructured. This is where text mining comes in, also referred to as Knowledge Discovery from Text(KDT).
``Text mining is the process of extracting interesting information and knowledge from unstructured text''\cite{hotho-etal-ldv-2005} and its applications tend to work in two steps, first using an Information Retrieval(IR) application to narrow the search space, and then they extract significant parts of the retrieved texts\cite{Polajnar2006}. This general process usually involves structuring a source text by means of parsing and other linguistic analysis, then finding patterns in this structured data and then interpreting this output.

Text mining is fundamentally different from standard web searching in that web searches rely primarily on information that is already known. However, the goal of text mining is to discover interesting, previously unknown information\cite{Gupta_Lehal_2009}.
There is however one key issue introduced by text mining; natural language is used by humans for communication and recording information, while computers are incapable of interpreting natural language. Humans are naturally able to find linguistic patterns in text and understand the semantics of what is being said. Computers, on the other hand, face difficulties in interpreting variations in written text through spelling, colloquialism and also the general context of the text. Nonetheless, computers have what humans do not, that is, computers are much more capable of processing large datasets at very high speeds, particularly in comparison to the human being. Thus, the objective of text mining is to combine the best of these both by creating an application that can retrieve relevant documents and then apply linguistic patterns which may be rule-based, using human-defined rules, or taught by means of machine learning techniques. This project takes the rule-based approach and as such only these techniques will be discussed.

An example of the text mining process can be seen in Figure~\ref{fig:tm}.
\begin{figure}[t]
\begin{center}
\includegraphics[width=15cm]{tm}
\end{center}
\caption{The text mining process\cite{Gupta_Lehal_2009}}
\label{fig:tm}
\end{figure}

\subsection[Information Retrieval]{Information Retrieval(IR)}
IR is the process of retrieving textual documents which may contain the answers to questions but do not answer these themselves\cite{hotho-etal-ldv-2005}. Information retrieval is fundamentally a web search working off user queries representing an information need. The process works by searching a collection of documents, and then retrieving those matching a user query depending on relevance. The approach to calculating relevance is dependent upon the actual IR engine itself, generally working on the frequency of specific key terms in each of these documents, and usually assigns a relevance rank to each document. This allows a sorting amongst the results and gives improved results, especially when given a limited number of results.

The IR tasks in this project will mainly be carried out on Twitter's systems, and as such, besides the core concept of IR, its internal specifics are not in the scope of this report.

\subsection[Natural Language Processing]{Natural Language Processing(NLP)}
NLP, in the scope of this project, is the process of extracting information from natural language\cite{Healey98}, that is, any language written or spoken by humans. This involves parsing and processing unstructured text to be able to gain meaningful knowledge from it. Nowadays most natural language processing is done using machine learning techniques, however, in the past implementations were based on large sets of coded `rules'. These rules are used to define certain linguistic features in the text in order to understand the semantics behind it. NLP is a major field of research at present and also has applications in both information retrieval and information extraction.

There are many methods involved in NLP tasks and some of these will now be further explored.

\subsubsection{Tokenisation}
Tokenisation is the process of splitting a stream of text into singular words or phrases, otherwise known as tokens. These tokens usually form the basis of further NLP work. While it can be a straightforward process when using Standard English, the definition of a word, from the tokeniser's point of view, can be somewhat ambiguous. This is particularly true when considering the use of apostraphes. Figure~\ref{fig:tokenisation} shows an example of the different ways of tokenising the word \emph{don't}.

\begin{figure}[h!]
  \centering
  \setlength{\unitlength}{0.0125in}
\begin{picture}(80,105)( -20, 0)
\thicklines
\put(0,100){\framebox{don't}}

\put(0,80){\framebox{dont}}

\put(0,60){\framebox{don}}
\put(30,60){\framebox{'t}}

\put(0,40){\framebox{don}}
\put(30,40){\framebox{t}}

\put(0,20){\framebox{do}}
\put(25,20){\framebox{n't}}

\put(0,0){\framebox{do}}
\put(25,0){\framebox{nt}}
\end{picture}

  \caption{The different ways of tokenising the word \emph{don't}
    \label{fig:tokenisation}}
\end{figure}

These variations can be problematic in terms of the results being output for certain user queries. For example, in the case of these differing tokenisations of \emph{don't}, a user search for the word \emph{don} would return true twice, but should be false in the actual context. The importance of normalisations is highlighted tokenising tweets because a lot of users do not use apostrophes, either due to ease when typing, or in order to reduce the number of characters being used. Thus, varying spellings of the terms should not be tokenised differently.

\subsubsection{Normalisation}
Once text has been tokenised, these words may need normalising. Normalisation accounts for the several variations in spelling. For example, if you want to search for \emph{Mozilla~Firefox} you would want an IR engine to return not only documents containing the exact query but also those containing terms such as \emph{Firefox}, \emph{firefox} or \emph{mozilla~firefox}. Not doing this would obviously yield fewer results, or in the case of information extraction, it may suggest that \emph{Mozilla~Firefox} and \emph{mozilla~firefox} are two different things. Thus, normalisation is required to successfully map equivalent classes of terms.

\subsubsection{Stop Words}
Stop words are very commonly used words like \emph{a}, \emph{and} or \emph{the}. By creating a list of these terms, a \emph{stop list}, a natural language parser can remove these terms from the source text as they hold little or no value in matching queries to documents. In modern systems, however, stop lists are not widely used as they provide little gain in terms of efficiency\cite{manning2008}.

\subsubsection{Part of Speech(POS) Tagging}

\subsubsection{Stemming and Lemmatisation}
Documents contain many different derivations of words, such as \emph{normalise} and \emph{normalisation}, and differing forms of the same word due to its usage or tense, for example, \emph{walked} or \emph{walking}. An information extraction tool should ideally see these as somewhat equivalent terms; this is where stemming and lemmatisation come in. The following example, taken from \cite{manning2008}, shows how these techniques can map text:

\begin{quote}
am, are, is \begin{math}\Rightarrow\end{math} be 
\newline
car, cars, car's, cars' \begin{math}\Rightarrow\end{math} car
\newline
\newline
the boy's cars are different colors  \begin{math}\Rightarrow \end{math}
\newline
the boy car be differ color
\end{quote}

While stemming is a heuristic process hoping to achieve this goal, lemmatisation utilises a more sophisticated approach in that it uses a vocabulary and morphological analysis of words.

\subsection[Information Extraction]{Information Extraction(IE)}

\subsubsection{Named Entity Recognition(NER)}

\section[Sentiment Analysis]{Sentiment Analysis and Opinion Mining}

\section{Twitter Mining}
There has been several previous works on text mining Twitter posts, however, the bulk of these have focussed primarily on biomedicine and the financial sector.


% MOVE TWITTER API HERE? FROM IMPLEMENTATION CHAPTER

\chapter{Design}
\label{cha:design}
This chapter details the overall design of the system to be developed in this project. The software engineering methodology to be used will first be discussed, along with use cases, requirements and the architecture of the system. This will be followed up with notes on the class and database design diagrams. The decisions made with regards to some design choices will also be discussed in more detail.

\section{Methodology}
The software engineering methodology used in developing an application can have many effects on its final outcome. The development of this system will be carried out using principles taken from continuous integration and agile methods such as feature-driven development. There is always a working code repository available for deployment, and all new features to be implemented are to be worked on in clones of said repository. Upon completion of these minor implementations, they are tested to ensure everything is working correctly and assuming there are no issues, the changes are merged into the base repository. This methodology assures developers that if any major issues arise due to recent changes, they will be able to discard all changes and restart if they feel debugging would be a longer process. This ultimately allows for a faster development cycle and provides rigorous testing throughout the implementation process. This development methodology also allows for frequently changing requirements which is to be expected in any development task.

\section{Use Cases}
\label{sec:uc}
There are two main use cases for the proposed system. The following use case definitions apply only to the first stage of the system, i.e. retrieving and storing posts from Twitter. In both cases the system requirements converge, and will be explored below.

\subsection[Use Case 1]{UC1 - Streaming Twitter}
The first use case for the system requires a tool capable of continuously monitoring public tweets and storing these in a database. These tweets should be filtered by language and relevance, that is, tweets should be related to software. These tweets need to be filtered by language to counter any issues faced in the feature extraction stage due to the complexities involved in NLP. This use case will thus be referred to as streaming Twitter as that is its principle aim.

\subsection[Use Case 2]{UC2 - Searching Twitter}
\label{sec:uc2}
The second use case for the proposed system requires the ability to search Twitter for tweets concerning user-specified key terms, and as such will henceforth be referred to as searching Twitter. These key terms should also be related to software. The returned tweets will also be filtered by language as in the streaming use case to counter the NLP complexities. An extra function required here should be to see which software tools have been mentioned most often, and these should be displayed to the users with the option to search again, or see the full analysis of these tools.


\subsection{Merging the Use Cases}
Upon fulfilling these core requirements, the systems should store these tweets in a database for work in the remaining stages. The first of these is extracting the previously stated features from these tweets. These features must again be stored separately from the initial tweet data, as some tweets may contain information regarding more than one piece of software. Finally, all of this information must be aggregated and shown to users in the form of charts displaying sentiment, and all relevant information found alongside it.

% MAY NEED UC DIAGRAMS

\section{General Architecture}
As previously explained, the system design follows a 3-stage approach, these being tweet retrieval, feature extraction and visualisation for users. These stages are shown in the general archicture model displayed in Figure~\ref{fig:general}. These will be further explored in Sections~\ref{sec:arc1}, \ref{sec:arc2} and \ref{sec:arc3} respectively.

\begin{figure}[h]
\begin{center}
\includegraphics[width=12cm]{design}
\end{center}
\caption{General architecture of the system}
\label{fig:general}
\end{figure}

\subsection{Retrieving Tweets}
\label{sec:arc1}
The first stage involves retrieving tweets from Twitter. The design for this stage follows the same concepts for each of the use cases defined. This can then be split further as seen in Figure~\ref{fig:phase1}. The program should retrieve a set of search terms from a dictionary along with some keywords that may be associated with software. These are to be used to form a request for tweets from Twitter. Twitter will respond with the corresponding tweets and data, which are to be checked for language, to ensure they are in English. The remaining set of English tweets are then to be further parsed to extract the required information for storing these tweets in the database.

\begin{figure}[h]
  \centering
  \setlength{\unitlength}{0.0125in}
\begin{picture}(300,35)(90,730)
\thicklines
\put(0,740){\framebox(75,20){Filter terms}}
\put(75,750){\vector( 1, 0){ 20}}
\put(95,740){\framebox(75,20){Twitter API}}
\put(170,750){\vector( 1, 0){ 20}}
\put(190,740){\framebox(110,20){Language detection}}
\put(300,750){\vector( 1, 0){ 20}}
\put(320,740){\framebox(80,20){Parse response}}
\put(400,750){\vector( 1, 0){ 20}}
\put(420,740){\framebox(70,20){Tweet DB}}
\end{picture}

  \caption{Design for tweet retrieval
    \label{fig:phase1}}
\end{figure}

This returned information will be stored in a relational database and its design is shown below in Figure~\ref{fig:db}. Tweets will not be stored alone but also with simple user information to allow for future user profiling for a more targetted approach to tweet retrieval. The actual tweet information to be stored is its id on the Twitter platform - allowing for cases where users delete their tweets - its text content, time of creation, user id and the keyword used to find it, where one was used. There will also be fields for sentiment, i.e. positive, negative or neutral, and sentiment strength, where weightings have been used. These fields should default to \texttt{NULL} as their values will be computed at a later time. Finally, there is a \emph{tagged} boolean flag which signifies the given tweet has been processed, and the feature extraction process has been carried out on it. This is to be implemented in MySQL due to its simplicity, compatibility with the design and also because it is readily available on the university computers where data can be easily accessed both locally and remotely.

\begin{figure}[h]
\begin{center}
\includegraphics[width=12cm]{db}
\end{center}
\caption{Database design for storing tweets}
\label{fig:db}
\end{figure}

\subsection{Feature Extraction}
\label{sec:arc2}
The feature extraction stage will execute the task of extracting information from tweets. This will involve the stages described in \ref{sec:textmining} and its general design can be seen in Figure~\ref{fig:phase2}. The main aim of this stage is to take the tweets previously stored in the database and find the softwares mentioned in them.

\begin{figure}[h]
  \centering
  \setlength{\unitlength}{0.0125in}
\begin{picture}(300,160)( 70, 40)
\thicklines
\put(20,180){\framebox(70,20){Tweet DB}}
\put(55, 180){\vector(0,-1){20}}
\put(0,140){\framebox(110,20){Sentiment analysis}}
\put(110,150){\vector( 1, 0){ 20}}
\put(130,140){\framebox(100,20){URL extraction}}
\put(230,150){\vector( 1, 0){ 20}}
\put(250,140){\framebox(80,20){Tokenisation}}
\put(330,150){\vector( 1, 0){20}}
\put(350,140){\framebox(100,20){Price extraction}}
\put(400,140){\line(0,-1){20}}
\put(400,120){\line(-1,0){310}}
\put(90,120){\vector( 0, -1){ 20}}
\put(50,80){\framebox(80,20){POS tagging}}
\put(130,90){\vector( 1, 0){ 20}}
\put(150,80){\framebox(140,20){Main feature extraction}}
\put(290,90){\vector( 1, 0){ 20}}
\put(310,80){\framebox(100,20){Verify software}}
\put(360,80){\vector(0,-1){20}}
\put(295, 40){\framebox(130,20){Extracted features DB}}
\end{picture}

  \caption{Design for feature extraction
    \label{fig:phase2}}
\end{figure}

Due to the volatile nature of the information being extracted from these tweets, this data should be stored in a NoSQL-type database, that is, a schema-less design that diverges from the traditional relational database model. As a result, there is no formal design for this database structure, however, it is required of the system to at this stage store any features it has managed to extract along with the key information associated with the tweet that had been stored in the tweet retrieval stage, such as its unique id and text content.

\subsection{Visualisation}
\label{sec:arc3}
The visualisation stage has the task of displaying all the gained information and knowledge to the user. Ultimately, it must also provide a graphical interface for users to interact with the system in order to perform their own searches.

There are two parts to this phase of project. Firstly, the data gathered must be \emph{aggregated} so that meaningful knowledge can ultimately be represented. Upon completing this task, a \emph{graphical user interface} (GUI) must be designed for users to interact with the system, perform their own queries, and look at what information has been found.

\subsubsection{Aggregation}
Aggregating the results is the process of bringing together all ofthe different data sources for data on a single output entity such as a piece of software. This aggregated data can then be used easily by the GUI to display understandable information to the user. In this project, this task will entail grouping the stored documents by software or other similar information such as company or operating system. Once this information has been retrieved, statistics and charts can be derived. The main task here is to create pie charts for each piece of software to show the sentiment of the tweets mentioning them. Other details include the most tweeted pieces of software.

\subsubsection[Graphical User Interface]{Graphical User Interface (GUI)}
Upon the completion of the aggregation tasks, these charts and statistics need to be passed to a GUI. The GUI of choice for this system is a web application, as opposed to a desktop application. This is a more extensible solution as changes would not need to be pushed out to all users.

There will be three actions for users to carry out on this GUI. These will relate to use case 2 of this system, as seen in Section~\ref{sec:uc2}. The first action to carry out will be searching for tweets posted on Twitter. A mockup of this action can be seen in Figure~\ref{fig:guiwire2}. Once these tweets have been retrieved from the Twitter Search API, they will be displayed on the web page, and in turn features will be extracted, and again displayed to the user.

After this, there will be an analysis page. This page will show a list of software or operating systems, from which a user can select to view charts and a list of tweets. These charts will have been creating in the aggregation stage, but should be dynamically created. This page is shown in Figure~\ref{fig:guiwire3}.

Finally, once a number of tools, that is, software, operating systems etc. have been found in tweets, the home page should display a list of the most tweeted of these items. Clicking on an item in this list should offer the option of searching again for it, in order to update information, or to view its charts and sentiment information. This should look as the wireframe in Figure~\ref{fig:guiwire1}.

\begin{figure}[h]
\begin{center}
\includegraphics[width=10cm]{guiwire1}
\end{center}
\caption{A mockup of the website's home page}
\label{fig:guiwire1}
\end{figure}

\begin{figure}[h]
\begin{center}
\includegraphics[width=10cm]{guiwire2}
\end{center}
\caption{A mockup of the website's search page}
\label{fig:guiwire2}
\end{figure}

\begin{figure}[h]
\begin{center}
\includegraphics[width=10cm]{guiwire3}
\end{center}
\caption{A mockup of the website's analysis page}
\label{fig:guiwire3}
\end{figure}


\chapter{Implementation}
\label{cha:impl}
This chapter outlines the main stages in implementating the designed system in code.

% software environment to be mentioned in each section

The system to be developed, as initially described in Figure~\ref{fig:general}, is fairly representative of a Question-Answering (QA) system. QAs are data mining systems which use Information Retrieval (IR) and Information Extraction (IE) techniques to answer user queries. As such, this project was implemented in 3 stages, each corresponding to these subsystems in QA systems. The first of these was retrieving tweets, the IR phase of this project. Upon successful retrieval, information, the required features in this case, must be extracted and finally, these results must be displayed to the user in a simple, straightforward fashion. These stages will be further explored as follows:

\java
\section{Tweet Retrieval}
Without tweets, there is no work to be done, and so retrieving tweets can be regarded as the most important part of this project. The main objective of this stage is to retrieve as many relevant tweets from Twitter as possible. To do this, the system will interact with the set of public APIs Twitter provides in order to fulfill the requirements of each use case stated in Section~\ref{sec:uc}. The system is designed to use all of these to fulfill the requirements of each use case. This subsystem in the project is implemented in Java. This is because of its strong object oriented nature and platform independency. Of course, there are other options such as Python, however Java is a programming language with relatively straightforward multithreading capabilities.

\subsection{Streaming Twitter}
The Streaming API allows the system to fulfill the requirements of having a fully automated system that constantly monitors Twitter for software-related posts, as described by use case 1, in Section~\ref{sec:uc1}. 

The implementation at this stage utilises Twitter's filtering URL at \url{https://stream.twitter.com/1/statuses/filter.json} and passes it the set of dictionary terms and keywords to filter tweets by described in Section~\ref{sec:arc2}.

This implementation could have been done using the \emph{Twitter4J}\footnote{\url{http://twitter4j.org/}} Java library for Twitter integration, however most of the functions appear unnecessary and excessive in the scope of this project. For this reason, the Twitter Streaming API integration was implemented from the ground up.

Upon retrieving these filter terms from the database, the application formats this list into a string after which it creates three new Thread objects, a \emph{DatabaseThread} which will carry out all database operations, a \emph{StreamParseThread} which parses the stream of responses sent back from the Twitter server, and a \emph{ScannerThread}, which monitors the running state of the program, so as to allow a clean exit when the user wishes to quit. This scanner thread simply monitors the console input for users to type the exit command, upon which all connections are dropped and final parsing and database operations are carried out before closing the application. This high level control flow can be seen in Figure~\ref{fig:tweetir}.

On initialising these threads, the application attempts to set up a secure connection to Twitter using the HTTPS protocol. It uses the POST method to write the string of filter terms to the server in order to being receiving tweets. Once this connection is fully set up, Twitter will return JavaScript Object Notation (JSON) strings for each tweet, and so a JSON parser is set up using the Google-Gson Java library~\cite{gson}. The aforementioned threads are then started as the actual streaming process now begins.

For every tweet returned by the API, the application adds this JSON response, as a \emph{JsonObject}, to a queue in the \emph{StreamParseThread} class, using the following simple method:
\begin{lstlisting}[caption=Adding tweets to a parse queue, label=lst:queue]
private final List<JsonObject> parseList = new ArrayList<JsonObject>();

public boolean addTask(final JsonObject object) {
    synchronized (parseList) {
        return parseList.add(object);
    } // synchronized
} // addTask(JsonObject)
\end{lstlisting}

The parser thread now assumes control of the processing to be done, while the main thread continues to add to this \emph{parseList} queue. The parser thread has the sole task of parsing the information in this JSON object into a more meaningful \emph{Tweet} object. This class' properties can be seen in Listing~\ref{lst:tweetuser}. To do this, the JSON object first needs to be checked if it represents a tweet delete entity, that is, a object containing the ``delete'' key signifying a user has deleted their tweet. In such a case, Twitter requests that applications honour the user's requests and delete this tweet. If otherwise the JSON object is actually a tweet, the program extracts the Twitter user's details to check their locale. In cases where this is not English, a \texttt{null} value is returned and the tweet is ignored. If this test passes, all the remaining properties described in the \emph{Tweet} and \emph{User} classes are extracted and returned as a single Tweet object.

This Tweet object is encapsulated in an \emph{InsertKeywordTask} object. This class is an implementation of the \emph{DatabaseTask} interface, which is used to perform the different database operations when used in conjunction with the different \emph{DatabaseConnector} classes. The hierarchy of these classes can be seen in Figures~\ref{fig:dbtask} and \ref{fig:dbcon} respectively. To further clarify, the \emph{DatabaseThread} constructor takes a \emph{DatabaseConnector} object as an argument. This allows for a more extensible system as different types of database management systems can be used with the application. It must be noted that in the current implementation, the \emph{DatabaseThread} class has been implemented at a similarly high level of abstraction, and as such an \emph{SQLThread} extends this class for use with a MySQL database. The \emph{SQLThread} initialises with a \emph{TweetSQLConnector} object, as it only operates in classes related to tweet retrieval. The DatabaseThread class maintains its own queue of \emph{DatabaseTask} objects as the parser thread, previously seen in Listing~\ref{lst:queue}.

\begin{figure}[t]
\begin{center}
\includegraphics[width=12cm]{tweetir}
\end{center}
\caption{Control flow for tweet retrieval subsystem}
\label{fig:tweetir}
\end{figure}

\begin{lstlisting}[caption=Tweet and User class properties, label=lst:tweetuser]
public class Tweet {
    private final long tweetId;
    private final String tweet;
    private final String createdAt;
    private final User user;
    private String keyword = null; // Only used when filtering
    ...
} // class Tweet

public final class User {
    private final long id;
    private final String username;
    ...
} // class User 
\end{lstlisting}

\begin{figure}[h]
\begin{center}
\includegraphics[width=7cm]{dbtask}
\end{center}
\caption{DatabaseTask class diagram}
\label{fig:dbtask}
\end{figure}

\begin{figure}[h]
\begin{center}
\includegraphics[width=9cm]{dbcon}
\end{center}
\caption{DatabaseConnector class diagram}
\label{fig:dbcon}
\end{figure}

Class design aside, once this \emph{InsertKeywordTask} object has been created, it is added to the queue in the \emph{DatabaseThread}. The respective implementations of the \emph{doTask()} method in each of the DatabaseTask classes will be performed as this queue is emptied. In the case of InsertKeywordTask, this is simply \texttt{db.insertTweet(t)}, where \texttt{db} is the TweetDatabaseConnector passed to it in the \texttt{doTask()} method, and \texttt{t} is the Tweet object it was initialised with.

This process is completed when the \emph{TweetDatabaseConnector} inserts the tweet into the database, in the current implementation using Java Database Connectivity (JDBC) to manipulate the MySQL server.

\subsection{Searching Twitter}
\label{sec:searchjava}
The Search API will now be used to handle use case 2 of the designed system, as described in Section~\ref{sec:uc2}. With the Search API not operating in real time, the realisation of this use case can afford to be a much simpler system. The levels of multithreading displayed in streaming Twitter will not be required, as interaction with the API is more of a serial communication as can be seen in Figure~\ref{fig:ucsearch}.

\begin{figure}[h]
\begin{center}
\includegraphics[width=13cm]{ucsearch}
\end{center}
\caption{Searching Twitter sequence diagram}
\label{fig:ucsearch}
\end{figure}

The implementation of the Twitter search use case utilises the Twitter Search API URL at \url{http://search.twitter.com/search.json}. The API also offers eXtensible Markup Language (XML) format responses, however, working with JSON allows consistency within the system. As with the implementation of the Twitter stream use case, the application begins with the initialision of a \emph{DatabaseThread} and is initially given a single keyword which will be searched for. This keyword is chosen by the end user. A simple HTTP GET request is then made to the above URL and Twitter then returns up to 1500 tweets from the last seven days corresponding to the search term.

The Search API returns a different JSON response to that of the Streaming API. Each JSON string contains an \texttt{iso\_language\_code}, and this will be used to filter tweets by language. Once the desired information has been extracted, i.e. the properties of the \emph{Tweet} class, the remaining operations are carried out just as they are in the realisation of the Twitter streaming use case, that is, the tweet is encapsulated in an \emph{InsertKeywordTask} object and this is added to a queue in the \emph{DatabaseThread} for the insert task to be carried out.

% SHOW JSON RESPONSE IN APPENDIX


\python
\section{Feature Extraction}
\label{sec:fe}
Once tweets have been retrieved from Twitter and stored in the MySQL database, they are now available for feature extaction, which can be regarded as the core stage in implementing the system. This subsystem involves using NLP techniques to extract the information described in Section~\ref{sec:arc2} from each tweet. This subsystem is implemented in Python 2.7 due to the raw power it possesses and also due to the decision to use the Natural Language Toolkit (NLTK). It was felt that Python's speed and text manipulation would allow for a better implementation of the system.

There are many steps involved in implementing the feature extraction. These are explored in order of execution.

\subsection{Sentiment Analysis}
Sentiment analysis was recognised as one of the key features to be extracted from the initial design stages (see Figure~\ref{fig:phase2}). It has been implemented using the Sentiment140 Bulk Classification Service API. 
The sentiment analysis of tweets is carried out before any of the other feature extraction work. As tweets have been retrieved and stored in the MySQL database, this part of the system selects the latest tweets, retrieving just the id and text, that have yet to be analysed and packs them into a JSON string object of the format:

\begin{verbatim}
{ "data": [ { "id": "1234", "text": "Google Chrome is awesome!" },
            { "id": "1235", "text": "Safari 5.0.2 is out now" },
            { "id": "1236", "text": "I really hate the new Firefox" } ] }
\end{verbatim}

This JSON string is then posted to the Twitter Sentiment API where classifications into the positive, negative and neutral classes are carried out by a Maximum Entropy classifier, using unigram and bigram features, trained with tweets containing emoticons. The internal specifics of a Maximum Entropy classifier, however, is not in the scope of this project.

Currently only 100 tweets are analysed at a time due to time constraints when users wish to run the program in real time. By using a small number, the data needing to be transferred is minimal and allows for a more interactive user experience.

Using the previous example, the data is returned by the server in the following format, with a polarity field added to each analysed tweet with values 0, 2 and 4 corresponding to negative, neutral and positive respectively.
\begin{verbatim}
{ "data": [ 
  { "id": "1234", "text": "Google Chrome is awesome!", "polarity": 4},
  { "id": "1235", "text": "Safari 5.0.2 is out now", "polarity": 2 },
  { "id": "1236", "text": "I really hate the new Firefox", "polarity": 0 } 
] }
\end{verbatim}

Upon receipt of this response, the JSON formatted string is parsed and the corresponding record for the tweet previously stored in the MySQL database is updated with new values for sentiment score and the actual sentiment, using polarity and its semantic meaning respectively.

\subsection{URL Extraction}
Before extracting context and semantics from tweets, any URLs mentioned are found and removed. Assuming the tweet is software-related, these URLs are quite likely to be links to the software, or further reviews. This task is done using NLTK's \texttt{regexp\_tokenize()} function with \url{http://}\verb/[^ ]+/ passed as the regular expression that finds URLs. If the tweet is later found not to contain any software, these URLs are discarded. Potential issues with this implementation could be that a user may post a URL without the preceding \url{http://} protocol prefix and these would not be found by this regular expression. However, Twitter automatically converts URLs to their \url{http://t.co/} domain and so this is resolved on the Twitter server side.

\subsection{Tokenisation}
After URLs have been extracted and removed from the source text, the tweet is tokenised to produce an array of all the terms in the tweet. The tokenisation process is also done using NLTK's \texttt{regexp\_tokenize()} function, passing it the regular expression - \verb/\w+([.,]\w+)*|\S+/. This expression returned superior results to alternation tokenisation functions provided by NLTK, such as \texttt{wordpunct\_tokenize()} as it was capable of finding numbers and currencies without splitting them. Using the above example,

\begin{quote}
I really hate the new Firefox
\end{quote}
this would be tokenised to the following:
\begin{quote}
[`I', `really', `hate', `the', `new', `Firefox']
\end{quote}

The following example shows a more complicated tokenisation process.
\begin{quote}
Norton Anti-Virus released for \$50 \#ripoff

\begin{math}\Rightarrow\end{math}
[`Norton', `Anti', `-Virus', `released', `for', `\$50', `\#ripoff']
\end{quote}

\subsection{Price Extraction}
\label{sec:price}
Continuing on from this tokenisation of the original source text, the current subsystem attempts to find prices in the array of terms. This is done using Python's built-in regular expression module, \texttt{re}. A number of regular expressions are used to define patterns denoting numbers, currencies and quantifiers like `hundred' and `thousand'. As the form of prices vary, for example in the case of mobile apps you might find `£0.59', `59p' or even `59 pence', these combinations of tokens may be split across two tokens in the array returned from the tokenisation process. For this reason, it is necessary to iterate over all items in the list of tokens while remembering the previous one. This obviously means a less efficient system, however, it has produced the best results in such variable conditions.

\subsection{Part-of-speech (POS) Tagging}
The POS tagger used by this system is taken from the NLTK modules and uses the \texttt{pos\_tag()} function which takes a tokenised sentence as its only argument. Continuing from the first example, this process tags as follows: 
\begin{quote}
[`I', `really', `hate', `the', `new', `Firefox']

\begin{math}\Rightarrow\end{math}
[(`I', `PRP'), (`really', `RB'), (`hate', `JJ'), (`the', `DT'), (`new', `JJ'), (`Firefox', `NNP')]

\begin{center}
\begin{tabular}{ l | l }
  \hline                        
  PRP & Pronoun \\
  RB & Adverb \\
  JJ & Adjective \\
  DT & Determiner \\
  NNP & Proper Noun \\
  \hline  
\end{tabular}
\end{center}
\end{quote}

\subsection{N-Grams}
N-grams are sequences of n tokens from a given source text. The implementation of creating n-grams in this project is done using the \texttt{nltk.util.ngrams()} function. This process starts by creating a five-gram of the tweet tokens. This means a sequence of five tokens will be created from the array of tokens. The system utilises a five-gram sequence due to potentially long software names, basing this on the na\"{\i}ve assumption that these names will not exceed five words. This will allow for improved extraction of software names in the next stage. Using the Firefox tweet as a running example, the outcome of this five-gram modelling process can be seen below.

\begin{quote}
[`I', `really', `hate', `the', `new', `Firefox']

\begin{math}\Rightarrow\end{math}

[
 ( (`I', `PRP'), (`really', `RB'), (`hate', `JJ'), (`the', `DT'), (`new', `JJ') ),

 ( (`really', `RB'), (`hate', `JJ'), (`the', `DT'), (`new', `JJ'), (`Firefox', `NNP') )
]
\end{quote}

\subsection{Main Feature Extraction}
This tagging process consists of the core functions of the proposed system. Its purpose is to extract all the features that have yet to be extracted, that is, software names and versions, companies, programming languages and operating systems. It also attempts to find any prices that may previously have been missed, and also has the task of discovering software that is not already in the dictionary.

Having created a set of five-gram sequences from the tweet, the application may now iterate through each of these in an attempt to find any information that has not yet been found. For each of these sequences, the program iterates through each POS-tagged token in the sequence. The tagging process then proceeds as follows: 
\newline
\begin{algorithmic}
\If {token is tagged as a noun}
    \If {token is in dictionary of software, companies, os, programming languages}
        \If {previous token not tagged as determiner or preposition}
            \State Feature has been found
        \EndIf
    \EndIf
\EndIf
\end{algorithmic}

This rule filters out linguistics of the form shown in Figure~\ref{fig:rule1}.
\begin{figure}[h!]
 \centering
  \setlength{\unitlength}{0.0125in}
\begin{picture}(200,10)( 20, 0)
\thicklines
\put(0,0){\framebox(100,20){DETERMINER}}

\put(110,0){\framebox(80,20){SOFTWARE}}

\put(200,0){\framebox(20,20){??}}
\end{picture}

  \caption{A linguistic filter
    \label{fig:rule1}}
\end{figure}
If however, these conditions fail, usually in the case where none of the tokens are in the dictionary of keywords used to retrieve these tweets, a regular expression is used to find clues to the presence of new software.

\begin{quote}
\verb~^download$|^get$~
\end{quote}

The above regular expression matches on the words \emph{download} or \emph{get}. This works on the basis that many tweets about software are usually posted to promote said software. This is generally done by urging others to download it, and that too by means of application stores like the App Store, or Google Play. This then allows the next token to be analysed to check if it is in fact a piece of software. This is done by checking that token's part of speech tag, and if it is a noun, the following tokens are also assessed in case the name of the software is longer than one word. This possible software name is then noted and kept aside for verification by web search as discussed in Section~\ref{sec:bing}. This regular expression can be applied to the five-gram in conjunction with others in order to maximise the number of features extracted. The following expression could be used to find an operating system.

\begin{quote}
\verb~^on$|^for$~
\end{quote}

By applying these together in the form displayed in Figure~\ref{fig:rule2}, the system may be able to determine the platform upon which a piece of software runs.

\begin{figure}[h!]
 \centering
  \setlength{\unitlength}{0.0125in}
\begin{picture}(200,20)( 20, 0)
\thicklines
\put(0,0){\framebox(60,20){Download}}

\put(70,0){\framebox(80,20){CANDIDATE}}

\put(160,0){\framebox(20,20){for}}

\put(190,0){\framebox(20,20){OS}}
\end{picture}

  \caption{A linguistic pattern to find software and the operating system it may run on
    \label{fig:rule2}}
\end{figure}

Once software has been found in the tweet, a search for its version number begins. This essentially works on the assumption that once software has been named, if a version number is to appear, it will be stated either immediately afterwards, or after the word \emph{version}, or some derivative, such as \emph{ver} or \emph{v}.

Finally, this section of the implementation does one more check for prices, this is mainly for free software, as the word \emph{free} would not be found at the price extraction stage explored in Section~\ref{sec:price}. 
\begin{quote}
\verb~^is$|^for$~
\newline
\verb~^free$~
\end{quote}

The above regular expressions are used similarly to the usage of the expression finding operating systems that software may run on. If the first expression is matched in the text, it is likely the next token is a price. In cases where the next token is \emph{free}, it will match the second regular expression, thus signifying the price of the software. This is a na\"{i}ve approach but in most cases the prices will already have been extracted.

This concludes the main feature extraction process leaving just the verification of new software names via a web search. Listings~\ref{lst:extr} and \ref{lst:extr2} show examples of the data structure storing these extracted features.

\begin{lstlisting}[caption=Example of some extracted features, label=lst:extr]
{
  "company_id" : "23",
  "company_name" : "apple",
  "os_id" : "9",
  "os_name" : "ios",
  "software_name" : "ios",
  "tweet" : "There are some new wallpaper in Apple iOS 5.1. Check them out!",
  "tweet_db_id" : "439640",
  "version" : "5.1",
  "sentiment": "positive"
}
\end{lstlisting}

\begin{lstlisting}[caption=Another example of some extracted features, label=lst:extr2]
{
  "price" : "free",
  "software_id" : "159",
  "software_name" : "moodle",
  "tweet" : "Training hundreds or thousands of staff?? If you want easy, affordable e-learning training then download moodle free http://t.co/beZgqaOd",
  "tweet_db_id" : "440065",
  "url" : "http://t.co/beZgqaOd"
}
\end{lstlisting}

\subsection{Software Verification}
\label{sec:bing}
The feature extraction subsystem may discover new software, and as such needs to verify these are actually pieces of software and not something else. To do this the program utilises the Microsoft Bing API which returns web search queries. As the main tagging process checks the dictionary for matching software names, and the tweet retrieval engine uses both the dictionary and a set of keywords, there will be some pieces of software mentioned in the tweets that are not in the dictionary. As a result, these will be flagged as possible software names, and then queried on the Microsoft Bing search engine with the keywords ``movie'', ``music'', and ``software game''. These keywords were selected on the basis that the initial search key terms retrieved many tweets referring to music and films. The implementation of the Bing API integration can be seen in Appendix~\ref{app:bing}.
\newline
\begin{algorithmic}
\Function {bing\_search}{bing, term}
    \State music = bing.search(term, music)
    \State movie = bing.search(term, movie)
    \State software = bing.search(term, software game)
    \newline
    \If {size(software) greater than size(movie) and size(music)}
        \If {references to software in title and description}
            \State \Return True
        \EndIf
    \EndIf
    \newline
    \State \Return False
\EndFunction
\end{algorithmic}
If the number of results for software associated with the searched term is greater than corresponding results for films and music, the results are checked for identifiers of software in their headings. Therefore if any of the results suggests the searched term is a piece of software, that is assumed true.

\section{Storing the Extracted Information}
As the information being extracted is temporarily stored in a Python dictionary variable, it is essentially in the form of a JSON string. The database design for storing this information is also in the form of a NoSQL database. For this reason, a document-based database system seems to be the best approach. By using MongoDB, it is easy to store the extracted information, as it is as straightforward as directly storing the string representation of this variable as a record in the database. Once this information has been successfully stored in MongoDB, the boolean \emph{tagged} flag is set to True for the corresponding tweet in the MySQL database.

\section[Visualisation]{Visualisation/Graphical User Interface(GUI)}
The final stage of the project is to aggregate and present the results to the user in a GUI.

\subsection{Aggregation}
The aggregation process has been implemented in Python because the MongoDB wrapper class has already been written and due to the simple mapping between Python dict types and MongoDB's stored documents.

This stage involes two processes. The first of these is finding the general sentiment towards a particular piece of software from all of the tweets mentioning said software, as in use case 3 (Section~\ref{sec:uc3}). This is a simple \emph{find} query and the command and syntax for carrying out this task in Python is equivalent to its MongoDB counterpart. The following command would retrieve documents containing the key-value pair of \texttt{software\_name=firefox}, given the collection, MongoDB's equivalent of a table in a relational database, is named \texttt{tagged\_tweets}.
\begin{quote} 
\begin{lstlisting}
db.tagged_tweets.find( { `software_name': `firefox' } )
\end{lstlisting}
\end{quote}

Upon retrieving these documents, the application extracts the distinct values from the \texttt{sentiment} field of each document, which can be \emph{positive}, \emph{negative} or \emph{neutral}. For each of these values that appears, its appearance count is calculated using a looping function and these counts are used to make charts in the web application, the topic of discussion in Section~\ref{sec:webapp}.

The second task, use case 4 (Section~\ref{sec:uc4}), is to find the top tweeted software and also includes operating systems in the final implementation. This is completed using a grouping function defined in the Python MongoDB driver module.

\begin{quote}
\begin{lstlisting}
def _group(self, key):
        return self.tags.group(key=[key],
                               condition={key:{`\$ne':None}},
                               initial={`count':0},
                               reduce=`function(obj,prev) { prev.count += 1;}')
\end{lstlisting}
\end{quote}

The above function returns a list of items grouped by a given \texttt{key}. The items in the list are also given a \texttt{count} value which is the number of times that software name will have appeared in the database collection. This list is then sorted in descending order and a slice of the top ten of these results are returned to be displayed to the user.

\subsection{Web Application}
\label{sec:webapp}
The web application is required to work with both the Python and the Java code in this project. As such, a static web page is far from the desired solution. This leaves the options of a Java web servlet or a Python web application. Both of these are good options, however, running a Python interpreter inside the Java Virtual Machine(JVM) is a much slower process than invoking Java classes from Python. This can be done using Jython, a Python implementation for Java that allows developers to both invoke Java from Python and vice versa \cite{Juneau:2010}. An alternative solution is to run shell commands directly through the Python interpreter in order to run the Java classes. This is the chosen route of action because no information needs to be passed between the two platforms that cannot be sent as command-line arguments when running the Java code.

Now that this has been decided, a Python web framework must be chosen for the development of this GUI. The CherryPy web framework was selected ahead of the likes of Django and Pylons due to its simplicity and pythonic interface \cite{cherrypy}. For an appealing web design that is easy to create, a templating language must be used to embed Python code. This will be done using the Mako template library\footnote{\url{http://www.makotemplates.org/}}. Mako is a leading templating library that is used by some major websites, such as \url{python.org} itself. Mako was chosen ahead of others such as Genshi, which is an XML-based templating engine, as its syntax is much more similar to that of normal Python code. Mako templates also support inheritance as can be seen in Listing~\ref{lst:mako} provided in Appendix~\ref{app:webapp}. HTML5, CSS and the JavaScript library jQuery make up the rest of the web development environment.

As designed, the web application has to carry out three functions. Users must be able to \textbf{search} Twitter, \textbf{analyse} stored results and \textbf{view} the top tweeted operating systems and software. The first of these tasks, the search, is carried out using the Java implementation of integrating Twitter's Search API. Python's \texttt{subprocess} module allows shell commands to be executed inside the Python interpreter. This feature is exploited to run the \\\texttt{uk.ac.manchester.cs.patelt9.twitter.SearchAPI} class, its implementation details explored in Section~\ref{sec:searchjava}, and check its output in the shell. This output is piped through to the web application and finally displayed on the web page. This is shown in Chapter~\ref{cha:results} when discussing the results of the project.

The user may then opt to follow through the entire feature extraction process. By restricting the search to a fairly small number of tweets, this process is heavy in terms of time consumption. By opting to do so, the user allows the application to check the sentiment of each of the tweets, and then follow through the feature extraction process described in Section~\ref{sec:fe}. At the end of this extraction, all the new information is returned in the form of a list, ready for display on the web page.

The next task is to show the user the results that have already been stored in the database. As per the design, this is done by showing the user a drop down menu consisting of all of the software that exists in the database. Upon selecting from the list, the application uses the aggregation methods described in the previous section to retrieve sentiment data. These are then used as input to the Google Chart Tools API, which creates pie charts displaying percentages for each distinct sentiment value.

Finally, the web application's home page shows a list of the top ten most tweet operating systems and software as links to the charts of these tools. This is again done using the aggregation methods set up in the previous stage of implementation.

\section{Testing}

\section{Summary}


\chapter{Results}
\label{cha:results}
This chapter details the results of the project, in terms of the accuracy of the feature extraction, along with views of the user interface.

\section{Feature Extraction}
The analysis of results is a semi-automated process carried out by checking which tweets were said to contain software and which did not. In total, 3268 tweets were retrieved from Twitter and stored in the MySQL database. These have all been through the feature extraction process, and 1168 of these are said to mention software or operating systems, leaving 2100 that do not. Of these tweets, roughly 10\%, i.e. 350, were manually tagged purely with software names, for the purpose of evaluating the system. Comparing these results produces the accuracy measures shown in Tables~\ref{tbl:truefalse} and \ref{tbl:measures}. While 350 tweets were tagged, the Table~\ref{tbl:truefalse} states a total of 367. This is because many tweets contain more than one software name.

A precision of 79\% is reasonable as this means that 79\% tweets were correctly found to have contained software. 68\% recall is fairly low as this means 32\% of the tweets retrieved from Twitter are actually irrelevant in the sense that they are actually unrelated to software. This could well be an issue with the keywords used, but also perhaps more likely due to the ambiguity associated with many software names, such as \emph{Blender}, which is a 3D computer graphics product as well as a kitchen appliance. The 84\% specificity suggests SWOT is quite successful at identifying negative software matches when processing tweets.

\begin{table}[h]
\begin{center}
\begin{tabular}{|c|c|c|c|c|}\hline
True Positives&True Negatives&False Positives&False Negatives&\textbf{Total}\\\hline
117&162&32&56&\textbf{367}\\\hline
\end{tabular}
\end{center}
\caption{Results from comparing manual and automated tagging}
\label{tbl:truefalse}
\end{table}

\begin{table}[h]
\begin{center}
\begin{tabular}{|c|c|c|c|}\hline
Precision&Recall&Specificity&\textbf{F-measure}\\\hline
0.79&0.68&0.84&\textbf{0.73}\\\hline
\end{tabular}
\end{center}
\caption{Precision, recall, specificity and F-measure}
\label{tbl:measures}
\end{table}

\section{GUI}
The GUI for this system has been implemented to the design requirements stated in Section~\ref{sec:guid}. All pages display a navigation bar allowing access to the Tweet Retrieval subsystem as well as the analysis pages. The home page displays the top ten tweeted software as found by the application. This can be seen in Figure~\ref{fig:gui1}. The elements in this list are clickable hyperlinks to their respective analysis pages.

\begin{figure}[h]
\begin{center}
\includegraphics[width=15cm]{gui1}
\end{center}
\caption{The web application's home page}
\label{fig:gui1}
\end{figure}

The search page asks users to enter a single query term as shown in Figure~\ref{fig:gui2}. This should ideally be the name of some software, operating system or programming language. After completing this search, the page displayed in Figure~\ref{fig:gui3} shows all retrieved tweets corresponding to the given query term, after being sent through the feature extraction process.

\begin{figure}[h]
\begin{center}
\includegraphics[width=15cm]{gui2}
\end{center}
\caption{The web application's search page}
\label{fig:gui2}
\end{figure}

\begin{figure}[h]
\begin{center}
\includegraphics[width=15cm]{gui3}
\end{center}
\caption{The web application's extracted features page}
\label{fig:gui3}
\end{figure}

The final requirement is to display the key information to the user. This involves displaying a pie chart representing the sentiment of tweets mentioning the selected software, and is shown in Figures~\ref{fig:gui4} and \ref{fig:gui5}.

\begin{figure}[h]
\begin{center}
\includegraphics[width=15cm]{gui4}
\end{center}
\caption{The web application's analysis dropdown menu}
\label{fig:gui4}
\end{figure}

\begin{figure}[h]
\begin{center}
\includegraphics[width=15cm]{gui5}
\end{center}
\caption{The web application's analysis page for a specified piece of software}
\label{fig:gui5}
\end{figure}

\chapter{Evaluation}
\label{cha:eval}
With implementation and analysis complete, one can now identify and evaluate the key successes and shortcomings of this project. These evaluations have been performed on the basis of the following questions and has been carried out by means of 5 independent user evaluations of the software.

\begin{itemize}
\item Does the SWOT system work?
\item Is the information found novel and interesting?
\item Is the system easy to use?
\end{itemize}

SWOT was shown to work to an accuracy of 73\% when measuring against software names. This is a relatively low performance rating as over larger sets of data the number of inaccurate results will be much greater. This is mainly down to the recall of the system. There are many situations where references to software have been missed in tweets. There is also a large set of tweets where they are in fact not related to software or explicity mention their names. However, this may be caused by the ambiguous nature of the terms in the dictionary and keyword database. Improving these results may require more linguistic rules to better analyse text.

The information found by the system is mainly sentiment data. While this may be interesting to certain parties, user opinion appears somewhat indifferent. Sentiment data would be interesting to the developers, and users looking for new software. However, while SWOT recognises software that is not present in the dictionary, these software are not explicitly flagged as new when they first appear in tweets. Also, unless a user specifically searches for a new piece of software, it is unlikely to appear in many tweets present in the database. This should only be an issue in the early stages after deployment, as over time, increasing numbers of tweets will contain this software particularly after it has been added to the dictionary and is used to filter on Twitter's Streaming API.  On the other hand, with more popular software, the sentiment information may not be particularly interesting to most users because the general perception towards it is likely to be common knowledge. However, SWOT can still be used to verify this information.

Finally, the system's usability is one of the key issues in this project. The web application has a very simple, straightforward interface. Some users found this to be more appealing as it looks tidy, however, others felt it was minimalistic, particularly the home page, though easy to use. Users also suggested the analysis page required more charts and extra information, such as the tweets themselves.


\chapter{Conclusions}
\label{cha:conclusion}
This chapter discusses the author's reflections on the success and conclusions of the project. Suggestions for further work are made and concludes with a summary of the report. 

\section{Reflections}
In terms of the overall progress made over the course of the year I feel this project has to some extent been a success. It has been a steep learning curve and on that basis some good results have been achieved. However, in absolute terms, I think I have made many mistakes in the way I went about working on the project over the year. This is down to a few key issues.

\textbf{Time management} can be seen as one of the overriding causes of the shortfalls of this project. There were times when progress had completely stalled due to minor issues with implementation.

I also feel as though my \textbf{preparation} may not have been sufficient, as I had overlooked a few key concepts when developing parts of the system. For example, I did not create a training set of data that would ultimately allow me to use a machine learning approach in my system and I think this may have resulted in a slightly lesser performing program, in terms of its accuracy.
% MORE ISSUES HERE

However, as previously stated, there have been major strides over the year. I feel I have a much greater grounding in the core object oriented design and programming principles. As well as this, I have developed skills in many new technologies and concepts. This was the first time I have had to develop a multithreaded desktop environment, and I had never previously worked with Python, JSON, MongoDB or the HTTPS protocol. As such I am happy with the general progress made in this project.

\section{Future Work}
%futurework - word frequencies




\nocite{*}
\bibliography{refs}             % this causes the references to be
                                % listed

\bibliographystyle{alpha}       % this determines the style in which
                                % the references are printed, other
                                % possible values are plain and abbrv
%% Appendices start here
\appendix
\chapter{Dictionary of Software and Keywords}
\label{app:dictionary}

\chapter{Bing API Integration}
\label{app:bing}
\lstinputlisting{bing.py}



\chapter{Web Application Python Code}
\label{app:webapp}
\lstinputlisting[caption=Web application implementation, label=lst:web]{__main__.py}

\lstinputlisting[language=HTML,morekeywords={def,block,include},caption=The base.html base class for all web pages, label=lst:mako]{base.html}

\end{document}
