Twitter is a massive corpus for text and opinion mining.

This project aims to develop a system that text mines Twitter posts to find software or software development tools that have been mentioned by its users and to discover the general sentiment of users towards these software.

This faces many challenges with respect to Natural Language Processing, particularly with the use of informal English.

The development of this system takes a four-stage process from design through to implementation. These stages are Information Retrieval in retrieving tweets from Twitter, Information Extraction in extracting certain features from tweets, before aggregating the data and finally creating a graphical user interface for users to interact with the system.

The features to be extracted include software names, versions, prices and sentiment. Tweets are retrieved from Twitter based upon matching filter terms taken from a dictionary of software and a set of keywords that may be associated with software. This is followed by the analysis of tweets to extract features, before being aggregated and displayed to the user.

The implementation was coded in Java and Python, utilising various public APIs and libraries.

The system scores an F-measure of 0.73 for extracting software names from tweets. This measurement was taken from a 10\% sampling of the stored data. 

There are areas for improvement in the project. Of course the accuracy rating could be improved, but also a more detailed user interface. The system should also aim to extract the purpose behind a user posting a tweet. This would provide more interesting information than the current implementation.

